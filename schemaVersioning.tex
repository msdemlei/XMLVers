\documentclass[11pt,a4paper]{ivoa}
\input tthdefs

\usepackage{listings}
\lstloadlanguages{sh,make,[latex]tex}
\lstset{flexiblecolumns=true,numberstyle=\small,numbers=left,
  identifierstyle=\texttt,defaultdialect=[latex]tex,language=tex}

\usepackage{todonotes}

\usepackage[utf8]{inputenc}

\definecolor{texcolor}{rgb}{0.4,0.1,0.1}
\newcommand{\texword}[1]{\texttt{\color{texcolor} #1}}

\iftth
  \newcommand{\BibTeX}{BibTeX}
\fi

\iftth
 \newcommand{\comicstuff}[1]{
    \begin{html}<span class="comic">#1</span>\end{html}}
\else
  \newcommand{\comicstuff}[1]{(HTML exclusive material)}
\fi
\customcss{custom.css}

\title{XML Schema Versioning Policies}

\ivoagroup{Standards and Processes}

\author[http://www.ivoa.net/cgi-bin/twiki/bin/view/IVOA/PaulHarrison]{Paul Harrison}
\author[http://www.ivoa.net/cgi-bin/twiki/bin/view/IVOA/MarkusDemleitner]{Markus Demleitner}
\author[http://www.ivoa.net/cgi-bin/twiki/bin/view/IVOA/BrianMajor]{Brian Major}
\author[http://www.ivoa.net/cgi-bin/twiki/bin/view/IVOA/PatDowler]{Pat Dowler}

\editor{Paul Harrison}

\previousversion{This is the first public release}
       

\begin{document}

\SVN$Rev: 2940 $
\SVN$Date: 2015-05-04 10:52:00 +0100 (Mon, 04 May 2015) $
\SVN$URL: https://volute.googlecode.com/svn/trunk/projects/ivoapub/ivoatexDoc/ivoatexDoc.tex $

\begin{abstract}
This note describes the recommended practice for the evolution of IVOA
standard XML schemata.  The criteria for deciding what might be considered major
and minor changes and the policies for dealing with each case are described.
\end{abstract}



\section*{Acknowledgments}

The content of this note is derived from discussions that occurred in a splinter
session at the June 2015 IVOA interoperability meeting in Sesto, Italy.

\section{Introduction}

Many of the standard protocols and data models developed by
the International Virtual Observatory Association (IVOA) have used XML
\citet{std:XML} for message or object serialization. The content of this XML has
been defined using XML schema definition language (XSD) \citet{std:XSD}. The
particular schema that has been associated with a standard 
is defined by the ``target namespace'' (hereafter refered to as simply ``the
namespace'') of the schema - the namespace identifier itself is a URI that
typically has a form that contains the version number of the standard. There
exist in many programming languages XML parsers that can use the XSD schema
automatically to do a strong check whether an instance of an XML
document conforms to the given schema. If the XML document does conform to the
structure defined by the schema then it is known as ``valid'' and even a small
deviation from the specified structure will mean that the XML instance is
``invalid''. This strong check of validity is extremely useful in the context of
interoperating services and clients within the VO as it guarantees that both
sides of an interaction will agree on the content of the document. To maintain
this behaviour, the conventional XSD practice is to give the schema a new
identity (namespace) if any changes are made to it so that both client and
server can agree on the exact version of a schema that they are using for
checking validity of an XML instance.

Once a VO service has been standardised there will typically be a growing number
of clients that are coded against the particular version of the
schema - any changes to the definition of the schema that is subsequently used
by a new version of the service to create instance documents will result in
immediate classification as ``invalid'' by the clients that have not themselves being
updated to use the new schema.
As explained above, in general this property is desireable for guaranteeing interoperability, but it does limit
the ability even to correct errors in the original schema definition without
causing distruption to the deployed clients. This note describes the
circumstances in which it is permissible to make changes to a schema and not
change its namespace.

\url{http://www.xfront.com/Versioning.pdf}


\section{Schema Versioning}
It is the case that two XML instances are formally regarded by XML
conventions as not equivalent if the only textual difference between them is
that the namespace declaration is not the same. We
can use the observation that simply removing the namespace definition from both instances
would make the instances both textually the same and equivalent in the strict
XML sense to suggest how changes to the schema definition without necessarily
needing to change the namespace.
\subsection{Minor changes}
\subsubsection{Determining if the changes are indeed minor}
\subsection{Indicating the version number}


\section{Schema Use}
The schemata that are used in the IVOA standard services and data models are
listed below
\begin{itemize}
  \item VOTable
  \item \ldots
\end{itemize}

As mentioned in the introduction the schemata are typically associated with a
particular version of a standard and so are typically included in an appendix of
the standard and in addition are located on the ivoa website
\url{http://www.ivoa.net/schema} to allow software authors easily to obtain the
latest version.
\subsection{Hosting the Schema on the IVOA web site}
Existing practice has used namespace URIs that correspond to a URL that points
to the location that the 




\subsection{Server use}


\subsection{Client use}
\subsection{}
\section{Summary of Recommendations}
This section summarizes the main recommendations contained within this document.

\begin{itemize}
\item Namespace URL should contain only the associated standard major number
(x.0).

\item Include a version attribute in the top level element to allow client
version discovery - this version number should include the full standard version
number including the minor version increment. - 1.1

\item Set the version attribute of the <schema> element in the XSD to be equal
to the full standard version including the minor version increment and document status - to
allow software writers to be clear about exactly which version of the schema
they are using 1.1-20150607-WD

\item Minor version changes should not break clients - i.e. in general they
should only add new elements/attributes, not remove formerly valid content -
correspondingly, this means that clients should quietly ignore things that they
do not know about in the XML - i.e. they should not automatically issue an error
if schema validation fails (though of course if they can determine that the
validation error is because of a missing required element then they should
issue an error).
\item In contrast servers must produce strictly valid documents, and service
validators must test strict validity against the relevant schema (discovered
from the namespace and the version element)
\end{itemize}




\appendix


\section{Changes from Previous Versions}

No previous versions yet.  
% these would be subsections "Changes from v. WD-..."
% Use itemize environments.


\bibliography{ivoatex/ivoabib}


\end{document}
