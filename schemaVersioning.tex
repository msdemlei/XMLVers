\documentclass[11pt,a4paper]{ivoa}
\input tthdefs

\usepackage{listings}
\lstloadlanguages{sh,make,[latex]tex}
\lstset{flexiblecolumns=true,numberstyle=\small,numbers=left,
  identifierstyle=\texttt,defaultdialect=[latex]tex,language=tex}

\usepackage{todonotes}

\usepackage[utf8]{inputenc}

\definecolor{texcolor}{rgb}{0.4,0.1,0.1}
\newcommand{\texword}[1]{\texttt{\color{texcolor} #1}}

\iftth
  \newcommand{\BibTeX}{BibTeX}
\fi

\iftth
 \newcommand{\comicstuff}[1]{
    \begin{html}<span class="comic">#1</span>\end{html}}
\else
  \newcommand{\comicstuff}[1]{(HTML exclusive material)}
\fi
\customcss{custom.css}

\title{XML Schema Versioning Policies}

\ivoagroup{Standards and Processes}

\author[http://www.ivoa.net/cgi-bin/twiki/bin/view/IVOA/PaulHarrison]{Paul Harrison}
\author[http://www.ivoa.net/cgi-bin/twiki/bin/view/IVOA/MarkusDemleitner]{Markus Demleitner}
\author[http://www.ivoa.net/cgi-bin/twiki/bin/view/IVOA/BrianMajor]{Brian Major}
\author[http://www.ivoa.net/cgi-bin/twiki/bin/view/IVOA/PatDowler]{Pat Dowler}

\editor{Paul Harrison}

\previousversion{This is the first public release}
       

\begin{document}

\SVN$Rev: 2940 $
\SVN$Date: 2015-05-04 10:52:00 +0100 (Mon, 04 May 2015) $
\SVN$URL: https://volute.googlecode.com/svn/trunk/projects/ivoapub/ivoatexDoc/ivoatexDoc.tex $

\begin{abstract}
This note describes the recommended practice for the evolution of IVOA
standard XML schemata.  The criteria for deciding what might be considered major
and minor changes and the policies for dealing with each case are described.
\end{abstract}


\section*{Acknowledgments}

The content of this note is derived from discussions that occurred in a splinter
session at the June 2015 IVOA interoperability meeting in Sesto, Italy.

\section{Introduction}

Many of the standard protocols and data models developed by
the International Virtual Observatory Association (IVOA) have used XML
\citet{std:XML} for message or object serialization. The content of this XML has
been defined using XML schema definition language (XSD) \citet{std:XSD}.  As
these are normative texts, attention to detail is very important, and being able to rigorously track changes to the documents is highly advantageous.  


As mandated by \citet{std:docSTD}, 

\url{http://www.xfront.com/Versioning.pdf}

\section{Schema Use}
\subsection{Server use}
\subsection{Client use}
\section{Schema Versioning}
\subsection{Minor changes}
\subsubsection{Determining if the changes are indeed minor}
\subsection{Indicating the version number}
\subse
\section{Hosting the Schema on the IVOA web site}
Existing practice has used namespace URIs that correspond to a URL that points
to the 
\section{Summary of Recommendations}
This section summarizes the main recommendations contained within this document.

\begin{itemize}
\item Namespace URL should contain only the associated standard major number
(x.0).

\item Include a version attribute in the top level element to allow client
version discovery - this version number should include the full standard version
number including the minor version increment. - 1.1

\item Set the version attribute of the <schema> element in the XSD to be equal
to the full standard version including the minor version increment and document status - to
allow software writers to be clear about exactly which version of the schema
they are using 1.1-20150607-WD

\item Minor version changes should not break clients - i.e. in general they
should only add new elements/attributes, not remove formerly valid content -
correspondingly, this means that clients should quietly ignore things that they
do not know about in the XML - i.e. they should not automatically issue an error
if schema validation fails (though of course if they can determine that the
validation error is because of a missing required element then they should
issue an error).
\item In contrast servers must produce strictly valid documents, and service
validators must test strict validity against the relevant schema (discovered
from the namespace and the version element)
\end{itemize}




\appendix


\section{Changes from Previous Versions}

No previous versions yet.  
% these would be subsections "Changes from v. WD-..."
% Use itemize environments.


\bibliography{ivoatex/ivoabib}


\end{document}
