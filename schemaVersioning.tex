\documentclass[10pt,a4paper]{ivoa}
\input tthdefs

\usepackage{color}
\usepackage{listings}
\lstloadlanguages{sh,make,[latex]tex,XML,XSLT}
\lstset{flexiblecolumns=true,numberstyle=\small,numbers=left,
  identifierstyle=\texttt}


\usepackage{color}
\definecolor{gray}{rgb}{0.4,0.4,0.4}
\definecolor{darkblue}{rgb}{0.0,0.0,0.6}
\definecolor{cyan}{rgb}{0.0,0.6,0.6}

\lstset{
  basicstyle=\ttfamily,
  columns=fullflexible,
  showstringspaces=false,
  commentstyle=\color{gray}\upshape
}

\lstdefinelanguage{XML}
{
  morestring=[b]",
  morestring=[s]{>}{<},
  morecomment=[s]{<?}{?>},
  stringstyle=\color{black},
  identifierstyle=\color{darkblue},
  keywordstyle=\color{cyan},
 % morekeywords={xmlns,version,type}% list your attributes here
}


\usepackage{todonotes}

\usepackage[utf8]{inputenc}

\definecolor{texcolor}{rgb}{0.4,0.1,0.1}
\newcommand{\texword}[1]{\texttt{\color{texcolor} #1}}

\iftth
  \newcommand{\BibTeX}{BibTeX}
\fi

\iftth
 \newcommand{\comicstuff}[1]{
    \begin{html}<span class="comic">#1</span>\end{html}}
\else
  \newcommand{\comicstuff}[1]{(HTML exclusive material)}
\fi
\customcss{custom.css}

\title{XML Schema Versioning Policies}

\ivoagroup{Standards and Processes}

\author[http://www.ivoa.net/cgi-bin/twiki/bin/view/IVOA/PaulHarrison]{Paul Harrison}
\author[http://www.ivoa.net/cgi-bin/twiki/bin/view/IVOA/MarkusDemleitner]{Markus Demleitner}
\author[http://www.ivoa.net/cgi-bin/twiki/bin/view/IVOA/BrianMajor]{Brian Major}
\author[http://www.ivoa.net/cgi-bin/twiki/bin/view/IVOA/PatDowler]{Pat Dowler}

\editor{Paul Harrison}

\previousversion{This is the first public release}
       
\begin{document}

\SVN$Rev$
\SVN$Date$
\SVN$URL$

\begin{abstract}
This note describes the recommended practice for the evolution of IVOA
standard XML schemata that are associated with IVOA standards.  The criteria for
deciding what might be considered major and minor changes and the policies for dealing with each case are described.
\end{abstract}



\section*{Acknowledgments}

The content of this note is derived from discussions that occurred in a splinter
session at the June 2015 IVOA interoperability meeting in Sesto, Italy.

\section{Introduction}

Many of the standard protocols and data models developed by
the International Virtual Observatory Association (IVOA) have used XML
\citep{std:XML} for message or object serialization. The structure of
these XML files has
usually been constrained using the XML schema definition language
\citep{std:XSD}, or XSD for short.
The particular schema that has been associated with a standard 
is defined by the ``target namespace'' (hereafter refered to as simply ``the
namespace'') of the schema - the namespace identifier itself is a URI that
typically has a form that contains the version number of the standard. There
exist in many programming languages XML parsers that can use the XSD schema
automatically to do a strong check whether an instance of an XML
document conforms to the given schema. If the XML document does conform to the
structure defined by the schema then it is known as ``valid'' and even a small
deviation from the specified structure will mean that the XML instance is
``invalid''. This strong check of validity is extremely useful in the context of
interoperating services and clients within the VO as it guarantees that both
sides of an interaction will agree on the structure of the document and
hence its interpretation. To maintain
this behaviour, the conventional XSD practice is to give the schema a new
identity (namespace) if any changes are made to it so that both client and
server can agree on the exact version of a schema that they are using for
checking validity of an XML instance.

Once a VO service has been standardised there will typically be a growing number
of clients that are coded against the particular version of the
schema - any changes to the definition of the schema that is subsequently used
by a new version of the service to create instance documents will result in
immediate classification as ``invalid'' by the clients that have not themselves being
updated to use the new schema.
As explained above, in general this property is desirable for guaranteeing interoperability, but it does limit
the ability even to correct errors in the original schema definition without
causing distruption to the deployed clients. 

A related problem occurs because to namespace-aware XML parsers, the
element name is a tuple of the namespace URI and the tag content.  For
instance, in the XML document

\begin{lstlisting}[language=XML]
<doc xmlns="http://example.com/1.0"/>
\end{lstlisting}

the full name of the element is conventionally written as
\verb|{http://example.com/1.0}doc|, and clients will expect this full
form, whether or not they perform schema validation.  As soon as the
namespace changes, any client expecting this element name will no longer
understand anything in the document.  Rather typically, however, the
client would still be able to make sense of the document as far as it is
relevant to the client if it ignored the namespace part of the element
name.  In consequence, many clients started discarding the namespace
part entirely, which then leads to new interoperability problems, for
instance, when elements from a different namespace are embedded within
such a document.  Hence it is desirable to limit namespace changes to
cases when legacy clients would definitely break.

This note describes the
circumstances in which it is permissible to make changes to a schema and not
change its namespace.

\url{http://www.xfront.com/Versioning.pdf}\todo{What's the meaning of
this URI?}


\section{Schema Versioning}
It is the case that two XML instances are formally regarded by XML
conventions as not equivalent if the only textual difference between them is
that the namespace declaration is not the same. We
can use the observation that simply removing the namespace definition from both instances
would make the instances both textually the same and equivalent in the strict
XML sense to inform the class of changes can be to the schema definition without
necessarily needing to change the namespace, and how a client should treat such
changes.
\subsection{Minor changes}
In general the class of changes that might be considered minor are those
which; \todo{Actually, I'd prefer to have a functional criterion here,
along the lines of: ``when legacy clients ignoring any additional
content keep functioning''}

\begin{itemize}
  \item Do not remove concepts (i.e. elements or attributes) from the old
  schema.
  \item Make any new concepts that are introduced optional.
\end{itemize}
Even with the restrictive conditions above it is still necessary that any
consumer of XML instance documents takes the approach that it does not do strict
schema validation against the version of the schema that it knows about, but
rather ignore everything that it does not understand. This approach is allowable
because any new concepts are optional even for consumers of the XML instance
that are aware of the latest version of the schema, and so clients cannot use
this information for fundamental changes in the behaviour of an IVOA protocol.
In other words the IVOA protocol remains backwards compatible and would only
warrant a ``point change'' in the standard version.

Conversely if the conditions above cannot be met by an evolution of an IVOA
standard then it is a indication that the standard is undergoing a ``major
change'' and backwards compatibility may be broken. Such a change would
involve both a change in the first part of the standard document's version
 number as well as a change in the version used in the schema
namespace.
\subsubsection{Determining if the changes are indeed minor}
Although the conditions outlined above for minor changes should generally be
strictly adhered to in the design of minor extensions to standards, there are
occasions where a new version of a standard might try to correct an error made
in a previous version. For example a certain construct could have been
schema valid which the should not have been allowed in a correctly authored
schema that expressed the intentions of the standard. In this case it is likely
the the correction would break the first of the above conditions, but hopefully
clients would not be written to expect the unintentionally valid construct and
would not be unduely impacted by the change. It would be up to the IVOA
Technical Coordination Group (TGC) to determine the likely impact of such a
change and whether the violation of the constraints for a minor change would be
allowable in such circumstances.

\subsection{Indicating the version number}

When allowing minor changes to the schema without changing the
namespace it becomes imperative that the the minor version number of the schema
is communicated via a means other than the namespace. It is important that
clients are able to know which version of the service was used to create an XML
instance so that they might invoke new optional parts of the service. It is
recommended that the instance documents have a required \xmlel{version}
attribute on the top level element. It should be noted that for services where
there might be several XML responses for different endpoints with different top
level elements, then they should each have a \xmlel{version} attribute.

\subsubsection{Indicating the schema document version}
It is of similar importance to software writers to know exactly the version of
the schema that they are using. In the XSD definition there is an appropriate
\xmlel{version} attribute on the top level \xmlel{<schema>} element that can be
used for this purpose.

\begin{lstlisting}[language=XML]
<xs:schema xmlns:xs="http://www.w3.org/2001/XMLSchema"
   targetNamespace="http://www.ivoa.net/xml/UWS/v1.0"
   xmlns:uws="http://www.ivoa.net/xml/UWS/v1.0"
   xmlns:xlink="http://www.w3.org/1999/xlink" 
   elementFormDefault="qualified"
   attributeFormDefault="unqualified"
   version="1.1-PR-20150626"
>
\end{lstlisting}

This version attribute is not used formally by the schema validation machinery
but can be used as desired here to indicate a precise version of the schema
given that the namespace is only used to differentiate between major revisions
of the schema.



\section{Hosting the Schema on the IVOA Web Site}
As mentioned in the introduction the schemata are typically associated with a
particular version of a standard and so are typically included in an appendix of
the standard. In addition are located on the ivoa website
\url{http://www.ivoa.net/schema} to allow software authors interactively to
obtain the latest version directly as a file.

The availability of the schemata via the IVOA web site also
allows XML parsers to be instructed to fetch the schema automatically when
validating an XML instance. This is done using the \xmlel{xsi:schemaLocation}
attribute which pairs a namespace with the URL of a schema that has definitions
in that namespace. Although there is no particular necessity to do so, it has
been standard practice to use namespaces that have a one-to-one correspondence
with a location on the IVOA web site. The advantage of this approach is that the
namespace is ``owned'' by the IVOA and there is therefore much less chance of
accidental clashes of namespaces.

As this note now recommends that there are potentially many minor versions of a
schema all in the same namespace this one-to-one correspondence between
namespace and URL needs to be broken if all versions of the XSD schema documents
are to be available to download. It is recommended that the URL that
corresponds to namespace be the location of the the latest (possibly minor)
version of schema, and that the older minor versions of the schema be moved to
be available at other locations on the IVOA web site.
\todo{We need to determine what would be a good URL strategy that is not too
burdensome on the IVOA document coordinator to implement - need to think up a
general scheme for finding old versions.}

\subsection{Client use of schema}
Using this strategy means that client software that simply uses validation with
the parser utilizing this schema location hint will always pick up the latest
version of the schema, which should mean that the instance document will always
be valid assuming that the server has created a strictly valid document. Note
that this still applies even if the server is using an older (minor) version of
the schema, as new versions are only allowed to add optional features to the
schema.

It is often the case that client software is written to use a local copy of the
schema to avoid the overhead of continually downloading the schema. In this case
it is possible that the XML instance document will not be valid against the
local schema if the server is working against a newer minor version of the
schema - as it could have added an new optional element into the XML instance.
It is for this reason that the client side of an IVOA service should be
forgiving in ignoring elements that it is not expecting. It is possible for the
client to recognise that the XML instance is valid against a newer version of
the schema by comparing the version attribute of the root element of the XML
instance against the version attribute in the local XSD schema document.

\section{Summary of Recommendations}
This section summarizes the main recommendations contained within this document.

\begin{itemize}
\item The namespace URL should contain only the associated standard major number
(x.0), and therefore does not change when a minor version increment occurs.
e.g.\ \texttt{http://www.ivoa.net/xml/UWS/v1.0}\todo{Markus believes the
``.0'' should not be present here.  Namespace URIs should always point
to major versions only.}

\item Include a \xmlel{version} attribute in the top level element to allow
client version discovery - this version number should include the full standard version
number including the minor version increment. -- 1.1

\item Set the \xmlel{version} attribute of the \xmlel{<schema>} element in the
XSD to be equal to the full standard version including the minor version
increment and document status and version - e.g.\ 1.1-WD-20150607

\item Minor version changes should not break clients - i.e. in general they
should only add new elements/attributes, not remove formerly valid content -
correspondingly, this means that clients should quietly ignore things that they
do not know about in the XML - i.e. they should not automatically issue an error
if schema validation fails (though of course if they can determine that the
validation error is because of a missing required element then they should
issue an error).
\item In contrast servers must produce strictly valid documents, and service
validators must test strict validity against the relevant schema (discovered
from the namespace and the version element).
\end{itemize}




\appendix


\section{Changes from Previous Versions}

No previous versions yet.  
% these would be subsections "Changes from v. WD-..."
% Use itemize environments.


\bibliography{ivoatex/ivoabib}


\end{document}
